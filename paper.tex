\documentclass{article}
\usepackage{hyperref}
\usepackage{tikz}
\usepackage{python}
\usepackage[%
    natbib=true,%
    backend=biber,%
    backref=true,%
    citestyle=authoryear-comp,%
    bibstyle=authoryear,%
    maxbibnames=24,%
    maxcitenames=2,%
]{biblatex}

\hypersetup{%
    bookmarks=true,
    colorlinks=true,
    pdfauthor={Luis Pedro Coelho},
    citecolor=pondgreen,
    urlcolor=toastedchilipowder,
    linkcolor=toastedchilipowder,
}
\newcommand*{\cpp}{{C\nolinebreak[4]\hspace{-.05em}\raisebox{.4ex}{\tiny\textbf{++}}}}
\let\code\texttt
\addbibresource{references.bib}

\title{Mahotas: Open source software for scriptable computer vision}
\author{Luis Pedro Coelho}

\begin{document}
\maketitle

\section*{Abstract}
Mahotas is a computer vision library for Python. It contains traditional image
processing functionality such as filtering and morphological operations as well
as more modern computer vision functions for feature computation, including
local features.

The interface is in Python, a dynamic programming language, which is very
appropriate for fast development, but the algorithms are implemented in \cpp{}
and are tuned for speed.

Mahotas is available under a liberal open source license (MIT License) and is
available from \url{http://github.com/luispedro/mahotas} or the Python Package
Index (\url{http://pypi.python.org/pypi/mahotas}).

\section{Introduction}

Mahotas is a computer vision library for the Python. It operates on numpy arrays,
but its inner loops are implemented in \cpp{} for speed and ease of
implementation.

\subsection{Functionality and Interface}

The interface is a procedural interface, with most functions working
independently of each other (there is code sharing at the implementation
level, but this is hidden from the user).

The main functionality is grouped into the following categories:

\begin{description}
\item[\textsc{surf}] Speeded-up Robust Features.
\item[Wavelet] Haar and Daubechies wavelets.
\item[features] Some feature descriptors. In particular, Haralick texture
features, Zernike moments, local binary patterns, threshold adjacency
statistics.
\item[morphological functions] Erosion and dilation, as well as some more
complex operations.
\item[watershed] seeded watershed.
\item[polygon operations] convex hull, polygon drawing.
\end{description}

There are a few interface conventions which apply to many functions.

When meaningful, a structuring element is used to define neighbourhoods or
adjacency relationships (morphological functions, in particular, use this
convention). Generally, the default is to use a $3 \times 3$ cross as the
default if no structuring filter is given (the exception to this rule is the
median filter, where the default is a $3 \times 3$ square).

Often, functions take an argument named \code{out} where the output will be
stored. This argument is often much more restricted in type. In particular, it
must often be a contiguous array.\footnote{Numpy supports non-contiguous
arrays, which are most often slices into other, larger, contiguous arrays
(e.g., given a $128 \times 128$ contiguous array, one can build a $64 \times
128$ non-contiguous array by taking every other row).} Since this is a
performance feature, it is natural that the interface is less flexible
(accessing a contiguous array is much more efficient than a non-contiguous
one).

\subsection{Implementation}

Mahotas is written in \cpp, but almost always, the user calls a Python function
which checks types and then calls the internal function. This is slightly
slower, but it is easier to develop this way.

The main reason that mahotas is implemented in \cpp{} (and not in C, which is
the language of the Python interpreted) is to use templates. Almost \cpp{}
functionality is split across 2~functions:

\begin{enumerate}
\item A \code{py\_function} which uses the Python C~API to get arguments and
check them.
\item A template \code{function<dtype>} which works for the type \code{dtype}
performing the actual operation.
\end{enumerate}

So, for example, this is how \emph{erode} is implemented. \code{py\_erode}
consists mostly of boiler-plate code:

\begin{cplusplus}
PyObject* py_erode(PyObject* self, PyObject* args) {
    PyArrayObject* array;
    PyArrayObject* Bc;
    if (!PyArg_ParseTuple(args,"OO", &array, &Bc)) {
        return NULL;
    }
    PyArrayObject* res_a = (PyArrayObject*)PyArray_SimpleNew(
                                array->nd,
                                array->dimensions,
                                PyArray_TYPE(array));
    if (!res_a) return NULL;
    PyArray_FILLWBYTE(res_a, 0);
    switch(PyArray_TYPE(array)) {
#define HANDLE(type) \
    erode<type>(numpy::aligned_array<type>(res_a), \
                numpy::aligned_array<type>(array), \
                numpy::aligned_array<type>(Bc));

        HANDLE_INTEGER_TYPES();
#undef HANDLE
    ...
\end{cplusplus}


This functions retrieves the arguments, performs some sanity checks, performs a
bit of initialisation, and finally, switches in the input type with the help of
the \code{HANDLE\_INTEGER\_TYPES()} macro, which call the right specialisation
of the template that does the actual work. In this example \code{erode}
implements (binary) erosion:

\begin{cplusplus}
template<typename T>
void erode(numpy::aligned_array<T> res,
            numpy::aligned_array<T> array,
            numpy::aligned_array<T> Bc) {
    gil_release nogil;
    const unsigned N = res.size();
    typename numpy::aligned_array<T>::iterator iter = array.begin();
    filter_iterator<T> filter(res.raw_array(), Bc.raw_array());
    const unsigned N2 = filter.size();
    T* rpos = res.data();

    for (int i = 0;
                i != N;
                ++i, ++rpos, filter.iterate_both(iter)) {
        for (int j = 0; j != N2; ++j) {
            T arr_val = false;
            filter.retrieve(iter, j, arr_val);
            if (filter[j] && !arr_val) {
                goto skip_this_one;
            }
        }
        *rpos = true;
        skip_this_one: continue;
    }
}
\end{cplusplus}

The template machinery makes the functions that use it very simple and easy to
read. The only downside is that there is some expansion of code size when the
compiler instanciates the function for the several integer and floating point
types. Given the small size of these functions, this is not a big issue.

In the snippet above, you can see some other \cpp{} machinery:

\begin{description}
\item[\code{gil\_release}] This is a ``resource-acquisition is object
initialisation'' (\textsc{raii})\footnote{\textsc{Raii} is a design pattern in
\cpp{}, or other languages with scope linked deterministic object destruction,
such as D, where a resource is represented by an object, whose constructor
acquires it and whose destructor releases it. This guarantees that the object
is correctly released even if the scope is left through an exception
\citep{Stroustrup1994}.} object that release the Python global interpreter lock
(\textsc{gil})\footnote{In the CPython interpreter, the most commonly used
implementation of Python, there is a global lock for many Python related
functionality, which limits parallelism.} in its constructor and gets it back
in its destructor. Normally, the template function will release the
\textsc{gil} after the Python-specific code is done. This allows several
mahotas functions to run concurrently.
\item[\code{array}] This is a thin wrapper around \code{PyArrayObject} that
knows its data type and has iterators which resemble the \cpp{} standard
library. This makes the code type-safer.
\item[\code{filter\_iterator}] This is taken from \code{scipy.ndimage} and it
is useful to iterate over an image and use a centered filter around each pixel
(it keeps track of all of the boundary conditions).
\end{description}

The inner loop is as direct an implementation of erosion as one would wish for:
for each pixel in the image, look at its neighbours. If all are true, then set
the corresponding output pixel to \textbf{true} (else, skip it as it has been
initialised to zero).

\section{Discussion}

Mahotas does not include machine learning related functionality, such as
$k$-means clustering or any classification. This is the result of an explicit
design decision. Specialised machine learning packages for Python (e.g.,
milk\footnote{http://github.com/luispedro/milk},
scikit-learn\footnote{http://}) already exist. A good classification system can
benefit both computer vision users and others. As these projects all use Numpy
arrays as their datatypes, it is easy to use functionality from the different
project seamlessly (including without any extra copying of data).

Python is an ideal language for fast development of both applications and
scientific software. With a fast library of computer vision and image
processing functions (implemented in \cpp{}, as the standard Python interpreter
is too slow for a direct Python implementation).

Mahotas has been available in the Python Package Index since April~2010 and has
been downloaded over 20,000~times. This does not include any downloads from
other sources. Mahotas includes a full test suite (version 1.0 has 100\% test
coverage). There are no known bugs.

\subsection*{Acknowledgements}

Mahotas includes code ported and incorporated from other projects. In
particular, the \textsc{surf} implementation is a port from the code from
\textit{dlib},\footnote{Dlib's webpage is at \url{http://dlib.net}.} a very
good \cpp{} library by Davis King. I also gleaned some insight into the
implementation of these features from Christopher Evan's OpenSURF library and
its documentation \citep{evans2009}.\footnote{OpenSURF is available at
\url{http://www.chrisevansdev.com/computer-vision-opensurf.html}, where several
documents describe details of the implementation.} The code which interfaces
with the FreeImage library, was written by Zachary Pincus and some of the
support code was written by Peter J. Verveer for the \code{scipy.ndimage}
project. All of these contributions were integrated while respecting the
software licenses under which the original code had been released. Robert Webb,
a summer student at Carnegie Mellon University, worked with me on the initial
local binary patterns implementation. Finally, I thank the several users who
have reported bugs, submitted small fixes, and participated on the project
mailing list.

\textbf{Funding}: I was supported in my work by the Funda\c c\~{a}o para a
Ci\^encia e Tecnologia (grants SFRH/BD/37535/2007 and PTDC/SAU-GMG/115652/2008)
and by a grant from the Siebel Scholars Foundation.

\printbibliography
\end{document}
